\chapter{Analyse}\label{chap:Analyse}

\section{Use Case Beschreibungen}

\subsection{Use Case "Spiel vorbereiten"}
\begin{labeling}[:]{Offene Punkte}
\item [Akteure] Spieler
\item [Priorität] Wichtig
\item [Beschreibung] Die Spieler wählen aus den vorhandenen Kategorien 4 aus. Die Fragen jeder Kategorie werden gemischt. Jeder Spieler wählt eine Farbe für sich aus.
\item [Vorbedigungen] Es existieren mindestens 4 Kategorien
\item [Offene Punkte] Was wird gemacht, wenn weniger als 4 Kategorien existieren.
\end{labeling}

\subsection{Use Case "Spiel starten"}
\begin{labeling}[:]{Offene Punkte}
\item [Akteure] Spieler
\item [Priorität] Essentiell
\item [Beschreibung] Der Spieler erhält initial seine 3 Wissensstreiter in der gewählten Farbe und platziert sie auf seinen Heimatfeldern. Zusätzlich erhält der Spieler 4 Wissensstandanzeiger, die jedoch noch nicht auf dem Spielfeld platziert werden.
\item [Vorbedigungen] Mindestens 2 Spieler vorhanden
\item [Offene Punkte]

\subsection{Use Case "Startspieler bestimmen"}
\begin{labeling}[:]{Offene Punkte}
\item [Akteure] Spieler
\item [Priorität] Essentiell
\item [Beschreibung] Die Spieler würfeln der Reihe nach einmal mit einem Würfel. Der Spieler, dessen Wurf die höchste Augenzahl aufweist, darf beginnen. Sollten mehrere Spieler die gleiche höchste Augenzahl gewürfelt haben, so müssen nur diese erneut untereinander auswürfeln, wer beginnt. Dies wird so lange gemacht, bis ein Spieler eindeutig als Startspieler ausgemacht ist.
\item [Vorbedigungen] Spiel gestartet
\item [Offene Punkte]

\subsection{Use Case "Spielzug machen"}
\begin{labeling}[:]{Offene Punkte}
\item [Akteure] Spieler
\item [Priorität] Essentiell
\item [Beschreibung] Hat ein Spieler alle 3 Wissensstreiter auf seinen Heimatfeldern, darf er in diesem Zug 3 Mal würfeln um einen Wissensstreiter ins Spiel zu bringen. Er bringt einen seiner Wissensstreiter ins Spiel, indem er eine 6 würfelt. Würfelt er innerhalb dieser 3 Würfe eine 6, so wird einer siner Wissensstreiter auf dem Pfad des Wissens platziert (auf dem Feld mit der Spielerfarbe) und sein Zug ist beendet. Sollte der Spieler nach den 3 Würfen keine 6 gewürfelt haben, so wird der Zug beendet und der nächste Spieler nach dem Uhrzeigersinn ist an der Reihe. \\
Hat ein Spieler mindestens einen seiner Wissensstreiter auf dem Pfad des Wissens, so würfelt er ein Mal. Würfelt er eine 6 und hat noch einen wissensstreiter auf seinem Heimatfeld, so muss er diesen ins Spiel bringen. Steht eine fremde Figur auf dem Startfeld für diesen Wissensstreiter, so beginnt automatisch eine Fragerunde. \\
Hat der Spieler keine weiteren Wissensstreiter auf seinen Heimatfeldern, so darf er einen seiner Wissensstreiter um die gewürftelte Augenzahl auf dem Pfad nach vorne setzen. Er darf jedoch keinen seiner Wissensstreiter auf ein Feld setzen, auf dem sich schon ein anderer Wissenstreiter von ihm befindet. \\
Kommt einer der Wissensstreiter auf ein Feld, auf dem bereits ein Wissenstreiter eines anderes Spieler steht, so beginnt eine Fragerunde mit diesem Mitspieler:

Der Spieler stellt seinem Mitspieler (auch "Geprüfter" genannt) eine Frage aus einer der 4 Kategorien. Kann der Geprüfte die Frage beantworten, so kann dieser seinen Wissensstandanzeiger für die gewählte Kategorie auf die nächst höhrere Stufe setzen. Ist die höchste Stufe bereits erreicht worden, so kann ein Wissensstandanzeiger in einer anderen Kategorie (die von dem Mitspieler bestimmt wird) erhöht werden. Im Anschluss wird der Wissenstreiter des Geprüften auf dessen Startfeld gesetzt. Befindet sich dort bereits ein Wissensstreiter des selben Spielers, so wandert der Wissensstreiter auf einem der Heimatfelder. \\
Sollte der Geprüfte die Frage jedoch nicht korrekt beantworten können, wird der Wissensstandanzeiger eine Stufe herabgesetzt und der Wissensstreiter landet auf einem der Heimatfelder. Im Anschluss muss der Fragestellende die selbe Frage beantworten. Wenn er sie richtig beantwortet, bleibt seine Spielfigur auf dem Feld stehen, ansonsten gelten die selben Regeln, wie bei dem Geprüften. \\

Hat nach dem Zug ein Spieler alle Wissensstandanzeiger auf der höchsten Stufe, so gewinnt dieser das Spiel und das Spiel ist beendet.
\item [Vorbedigungen] Vorheriger Spielzug ist beendet worden. Bei erstem Spielzug: Startspieler bestimmt.
\item [Offene Punkte]
\end{labeling}end{labeling}

\subsection{Use Case "Kategorie hinzufügen"}
\begin{labeling}[:]{Offene Punkte}
\item [Akteure] Spieler
\item [Priorität] Unwichtig
\item [Beschreibung] Der Spieler kann eine eigene Kategorie hinzufügen, der dann Fragen zugeordnet werden können.
\item [Vorbedigungen]
\item [Offene Punkte]
\end{labeling}end{labeling}

\subsection{Use Case "Frage hinzufügen"}
\begin{labeling}[:]{Offene Punkte}
\item [Akteure] Spieler
\item [Priorität] Unwichtig
\item [Beschreibung] Der Spieler kann eine Frage erstellen und mit einer Antwort versehen. Dann wird diese Frage einer Kategorie zugeordnet.
\item [Vorbedigungen] Kategorie vorhanden
\item [Offene Punkte]
\end{labeling}end{labeling}

\subsection{Begründung für die Priorisierung}

Die letzten beiden Punkte wurden als \texttt{unwichtig} eingestuft, da diese das Spielerlebnis zwar verbessern und personalisieren würden, jedoch ein Spiel ohne diese Funktionalität voll spielbar ist.\\
Das Spiel kommt nicht ohne die als \texttt{essentiell} gekennzeichneten Use Cases aus, da es ohne diese unspielbar ist.\\
Die Spielvorbereitung ist wichtig, da hierüber das Spielgeschehen maßgeblich gesteuert wird, also wie viele Spieler, mit welchen Kategorien spielen, aber sind nicht essentiell, da man hier auch mit festen Werten ein spielbares Spiel entwickeln könnte, was jedoch wenig Freude bereiten würde.

